\title{Lab report on subjects 005--008}
\author{
        John~O.~Woods,~Ph.D. \\
            \and
        Amie~E.~Hinds,~M.S. \\
            \and
        Luke Shafer,~M.S. \\
        R.~Daneel~Olivaw~Corporation\\
        908 Winston St., Suite \textsc{g} \\
        Houston, Texas 77009
}
\date{\today}

\documentclass[10pt]{article}

\usepackage{natbib}

\begin{document}
\maketitle

%\begin{abstract}
%This is the paper's abstract \ldots
%\end{abstract}

\section{Description}

Subjects 007 and 008 closely resemble subjects 001 and 002.

% Need descriptions of subjects.

\section{Methods and Results}
As always, we have limited our methods such that no intervention on our part was more extreme than the natural forces to which the objects were likely subjected prior to discovery, though we can only speculate about the time the objects have been present in our solar system (and their approximate locations and exposures during that time).

\subsection{Basic physical characteristics}

Given the behavior of the previously examined objects, we expected to be unable to measure the mass of the current subjects.
However, we observed constant mass --- or, at least, apparently constant across the observation times.

\begin{tabular}[h]{r|rrrr|r}
Measurement $t$ & 1:02 \textsc{pm} & 2:30 \textsc{pm} & 3:49 \textsc{pm} & 4:59 \textsc{pm} & mean \\
\hline
Subj.~003 & 0.241 & 0.240 & 0.241 & 0.242 & 0.241 kg \\
Subj.~004 & 0.227 & 0.225 & 0.226 & 0.226 & 0.226 kg \\
\hline
\end{tabular}

As before, the subjects were oddly buoyant in water.
We did not attempt a volume measurement, given the past behavior of similar subjects when submerged.
However, we measured the subjects' mass in combination with the water from the buoyancy test, taring the scale, and observed that the subjects now had the \textit{same} mass --- and a great deal less than the mass without the water.

\begin{tabular}[h]{r|rrr|r}
Measurement $t$ & 2:31 \textsc{pm} & 3:51 \textsc{pm} & 4:59 \textsc{pm} & mean \\
\hline
Subj.~003 & 0.012 & 0.011 & 0.010 & 0.011 kg \\
Subj.~004 & 0.011 & 0.011 & 0.012 & 0.011 kg \\
\hline
\end{tabular}

We observed, fire extinguisher at the ready, that the subjects were similarly buoyant in vegetable oil, but performed the test in a fume hood and thus did not measure the mass.

Since undertaking the tests on subjects 001 and 002, we have struggled to identify explanations for the weight oddities which satisfy Ockham's razor.
So far, the simplest explanation is one which seems utterly absurd to us, and lies outside our society's current understanding of physics.
The subjects seem to have properties and behaviors which we cannot describe in any other way than \textit{anti-gravitational} --- and could explain why there are no reports of such subjects being lost, \textit{e.g.} at sea.

We tested this hypothesis using what we call a \textit{two buckets of sand test}.

That is, we placed subject 003 at the bottom of a bucket, and poured a bucket of sand on top of it.
When we poured slowly, the subject exhibited an upward force as the bucket filled with sand, such that it remained atop the sand.
We repeated the experiment with subject 004, but this time poured the sand all at once.
The subject's change in weight appeared to be more or less immediate; the sand flowed around it until the subject again rested atop the sand.

We are reluctant to report the results of the next tests, because they bordered on absurdity.
We repeated the bucket test with a bucket of stale Trail's End buttered pre-popped popcorn (subject 003 again), and rather than the subject floating to the top, the popcorn seemed to blow out of the bucket.
A few kernels remained, but none on top of the subject.

Given the results of the bucket of popcorn experiment, and due to the exothermic behavior of the previous subjects when we attempted to submerge them, a clever intern, Marielle, suggested using \textit{un}popped popcorn.
We sent her to the store, and designed a setup wherein a glass candy thermometer (held using an oven mitt) would be used to hold the subject at the bottom of the bucket.
When Marielle returned with the corn kernels, we poured them quickly over the subject, while applying pressure to the thermometer.
The popcorn began to pop almost immediately --- within a second or two --- and then to smoke.
We quickly withdrew the thermometer, which registered a temperature above the limit of the thermometer ($400^\circ$ F).
Almost comically, the subject rose out of the smoldering kernels, bringing with it a few popped and burnt pieces, which flew out of the bucket.

% Figure: Mass of bucket of sand while object is anti-gravitating?

Further testing was prevented by time constraints.

\subsection{Electromagnetic properties}

The subjects possessed no magnetic field, and a magnet taken from an old hard drive seem to exert no magnetic force upon them (which we measured using a magnet suspended from a spring scale; the magnet's apparent weight did not change observably).

The subjects also do not appear to produce an electromagnetic field while anti-gravitation is in progress, suggesting that actuation is not through any electromagnetic means.

We attempted to take measurements using a cheap digital multimeter with sensor wires applied to various contacts on the two subjects.
First, we verified that the contacts were conductors, by applying the wires at the same time to the same contact; we measured a resistance of 0 $\Omega$.
We next attempted to apply the multimeter wires to adjacent contacts, but were unable to find any that would pass a current.
As each subject possessed 232 contacts, we were unable to systematically investigate all of the 53,592 combinations of contacts.

It is possible that the subjects' electrical internals have been destroyed, but it seems strange that devices capable of non-electromagnetically-induced anti-gravitation should have electrical internals that could so easily have been damaged.
It is also possible that the contacts are connected in triples rather than pairs --- such as with transistors or vacuum tubes --- or that they function only in some specific combination.
It is not clear why devices which appear not to rely on electromagnetism even need electrical contacts; perhaps they are something else entirely.

The black portions of the subject do not appear to conduct; neither do the polished stones, nor the apparent metallic ends on subject 004.

Obviously, whether an object possesses conductance or not is relative rather than absolute, but we were reluctant to apply higher voltages to the subjects.

\subsection{Radioactivity}
The subjects do not appear to possess radioactivity remarkably above background (alpha, gamma, beta).

\subsection{Spectroscopy}


\section{Conclusions}\label{conclusions}
%The object appears to possess some degree of situational awareness, based on its behavior in water and in the presence of a magnetometer.
%The four hours of study we were permitted on the object were insufficient for fully interrogating the range of possibilities produced by our testing, and the tests we performed have produced more questions than answers.

%We urge extreme caution when transporting the object due to its unpredictable behavior in a gravitational field.
%It is unclear how the artifact would behave, for example, in orbit --- or on the way to orbit.
%Similarly, the rapid temperature increase could pose a serious fire hazard, and we lacked sufficient time to determine if submersion in other materials would also produce rapid heating.

%\bibliographystyle{plainnat}
%\bibliography{03}


\end{document}
This is never printed