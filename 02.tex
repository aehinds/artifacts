\title{Lab report on subject 002}
\author{
        John~O.~Woods,~Ph.D. \\
            \and
        Amie~E.~Hinds,~M.S. \\
            \and
        Luke Shafer,~M.S. \\
        R.~Daneel~Olivaw~Corporation\\
        908 Winston St., Suite \textsc{g} \\
        Houston, Texas 77009
}
\date{\today}

\documentclass[10pt]{article}

\begin{document}
\maketitle

%\begin{abstract}
%This is the paper's abstract \ldots
%\end{abstract}

\section{Description}

Subject consists of two parts, which we refer to as the gem and the setting. The purpose is unknown. As with the first subject, the two components appear inseparable, limiting our options.
% measurements
% precise description of the stone
% precise description of the setting

Whereas subject 001 gem was transparent yellow, resembling lime citrine, this gem has the appearance of quartz, with rutilation, and appears to have been cut precisely.
% blue topaz
The setting is metallic, with a layer of what may be either oxidation or some sort of protective coating; the oxidization or coating has been scraped off on the setting's edges.

\section{Methods and Results}
As before, we endeavored to obtain as much information about the object as possible with as little potential impact on it as was feasible.
Throughout our analysis, we limited our methods such that no intervention on our part was more extreme than the natural forces to which the object was likely subjected prior to discovery.

\subsection{Basic physical characteristics}

As with the first object, measurements of the object's mass and volume were not obtainable in any straightforward way.
Prior experience suggested a need for more stringent safety protocols --- particularly the third-degree burns obtained by one analyst while trying to submerge the subject in water --- and these environmental health and safety concerns slowed our analysis significantly.

As before, we observed the presence of a strong magnetic field which fell off more quickly than expected as $r_\textrm{magnetometer}$ increased.
Again, the magnetic field appeared to rotate; vector measurements of the magnetic field using a survey magnetometer indicated that the north--south poles of the magnet were always aligned perpendicularly to a line segment connecting the subject and the sensor.
Rolling the magnetometer also caused the north--south poles to roll.
We observed no delay between the magnetometer's change in attitude and the change in the magnetic field.

Again, the apparent weight of the subject changed with the subject's orientation.
Periodic tests of the mass yielded a mean of 48.3 grams.
The general trend in mass was upward, and a least squares fit suggested a rate of 1.12 grams per hour.
Clearly, the subject's mass cannot increase indefinitely, and must come back down at some point; but four hours were not enough time to observe any downward trend.
The subject's apparent ability to violate the conservation of mass suggests either some sort of rapid interconversion between mass and energy, or favors an alternate explanation that seems straight out of a work of science fiction --- an explanation far too speculative to bear additional consideration here.

We repeated the centrifuge experiment from the last analysis, and this time observed that the force necessary to change the direction of the subject was constant, and suggested a mass of around 70 grams.
In this respect, today's subject differed from yesterday's.

As before, the subject floated in water.
Instead of trying to push it down with a paperclip, we used a plastic rod.
The rod began to melt as soon as the subject was submerged.
We were unable to measure a volume due to rapid phase change of the water to steam.

\subsection{Radioactivity}
Our observations consistently showed the object to be unremarkable in terms of its alpha, beta, and gamma emissions.

\subsection{Spectroscopy}
Spectroscopic analysis indicated that the gem is not, in fact, quartz.
% Need Luke and Amie's input here.
% Also setting.

\subsection{

\section{Conclusions}\label{conclusions}
The object appears to possess some degree of situational awareness, based on its behavior in water and in the presence of a magnetometer.
The four hours of study we were permitted on the object were insufficient for fully interrogating the range of possibilities produced by our testing, and the tests we performed have produced more questions than answers.

We urge extreme caution when transporting the object due to its unpredictable behavior in a gravitational field.
It is unclear how the artifact would behave, for example, in orbit --- or on the way to orbit.
Similarly, the rapid temperature increase could pose a serious fire hazard, and we lacked sufficient time to determine if submersion in other materials would also produce rapid heating.

%\bibliographystyle{abbrv}
%\bibliography{main}

\end{document}
This is never printed